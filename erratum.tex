%\documentclass{article}
\documentclass[12pt]{article}
\usepackage{latexsym}
\usepackage{amsmath}
\usepackage{amssymb}
\usepackage{relsize}
\usepackage{geometry}
\geometry{letterpaper}

%\usepackage{showlabels}

\textwidth = 6.0 in
\textheight = 8.5 in
\oddsidemargin = 0.0 in
\evensidemargin = 0.0 in
\topmargin = 0.2 in
\headheight = 0.0 in
\headsep = 0.0 in
%\parskip = 0.05in
\parindent = 0.35in


%% common definitions
\def\stackunder#1#2{\mathrel{\mathop{#2}\limits_{#1}}}
\def\beqn{\begin{eqnarray}}
\def\eeqn{\end{eqnarray}}
\def\nn{\nonumber}
\def\baselinestretch{1.1}
\def\beq{\begin{equation}}
\def\eeq{\end{equation}}
\def\ba{\beq\new\begin{array}{c}}
\def\ea{\end{array}\eeq}
\def\be{\ba}
\def\ee{\ea}
\def\stackreb#1#2{\mathrel{\mathop{#2}\limits_{#1}}}
\def\Tr{{\rm Tr}}
\newcommand{\gsim}{\lower.7ex\hbox{$
\;\stackrel{\textstyle>}{\sim}\;$}}
\newcommand{\lsim}{\lower.7ex\hbox{$
\;\stackrel{\textstyle<}{\sim}\;$}}
\newcommand{\nfour}{${\mathcal N}=4$ }
\newcommand{\ntwo}{${\mathcal N}=2$ }
\newcommand{\ntwon}{${\mathcal N}=2$}
\newcommand{\ntwot}{${\mathcal N}= \left(2,2\right) $ }
\newcommand{\ntwoo}{${\mathcal N}= \left(0,2\right) $ }
\newcommand{\none}{${\mathcal N}=1$ }
\newcommand{\nonen}{${\mathcal N}=1$}
\newcommand{\vp}{\varphi}
\newcommand{\pt}{\partial}
\newcommand{\ve}{\varepsilon}
\newcommand{\gs}{g^{2}}
\newcommand{\qt}{\tilde q}
\renewcommand{\theequation}{\thesection.\arabic{equation}}

%%
\newcommand{\p}{\partial}
\newcommand{\wt}{\widetilde}
\newcommand{\ov}{\overline}
\newcommand{\mc}[1]{\mathcal{#1}}
\newcommand{\md}{\mathcal{D}}

\newcommand{\GeV}{{\rm GeV}}
\newcommand{\eV}{{\rm eV}}
\newcommand{\Heff}{{\mathcal{H}_{\rm eff}}}
\newcommand{\Leff}{{\mathcal{L}_{\rm eff}}}
\newcommand{\el}{{\rm EM}}
\newcommand{\uflavor}{\mathbf{1}_{\rm flavor}}
\newcommand{\lgr}{\left\lgroup}
\newcommand{\rgr}{\right\rgroup}

\newcommand{\Mpl}{M_{\rm Pl}}
\newcommand{\suc}{{{\rm SU}_{\rm C}(3)}}
\newcommand{\sul}{{{\rm SU}_{\rm L}(2)}}
\newcommand{\sutw}{{\rm SU}(2)}
\newcommand{\suth}{{\rm SU}(3)}
\newcommand{\ue}{{\rm U}(1)}
%%%%%%%%%%%%%%%%%%%%%%%%%%%%%%%%%%%%%%%
%  Slash character...
\def\slashed#1{\setbox0=\hbox{$#1$}             % set a box for #1
   \dimen0=\wd0                                 % and get its size
   \setbox1=\hbox{/} \dimen1=\wd1               % get size of /
   \ifdim\dimen0>\dimen1                        % #1 is bigger
      \rlap{\hbox to \dimen0{\hfil/\hfil}}      % so center / in box
      #1                                        % and print #1
   \else                                        % / is bigger
      \rlap{\hbox to \dimen1{\hfil$#1$\hfil}}   % so center #1
      /                                         % and print /
   \fi}                                        %

%%EXAMPLE:  $\slashed{E}$ or $\slashed{E}_{t}$

%%

\newcommand{\LN}{\Lambda_\text{SU($N$)}}
\newcommand{\sunu}{{\rm SU($N$) $\times$ U(1)} }
\newcommand{\sunun}{{\rm SU($N$) $\times$ U(1)}}
\def\cfl {$\text{SU($N$)}_{\rm C+F}$ }
\newcommand{\mUp}{m_{\rm U(1)}^{+}}
\newcommand{\mUm}{m_{\rm U(1)}^{-}}
\newcommand{\mNp}{m_\text{SU($N$)}^{+}}
\newcommand{\mNm}{m_\text{SU($N$)}^{-}}
\newcommand{\AU}{\mc{A}^{\rm U(1)}}
\newcommand{\AN}{\mc{A}^\text{SU($N$)}}
\newcommand{\aU}{a^{\rm U(1)}}
\newcommand{\aN}{a^\text{SU($N$)}}
\newcommand{\baU}{\ov{a}{}^{\rm U(1)}}
\newcommand{\baN}{\ov{a}{}^\text{SU($N$)}}
\newcommand{\lU}{\lambda^{\rm U(1)}}
\newcommand{\lN}{\lambda^\text{SU($N$)}}
%\newcommand{\Tr}{{\rm Tr\,}}
\newcommand{\bxir}{\ov{\xi}{}_R}
\newcommand{\bxil}{\ov{\xi}{}_L}
\newcommand{\xir}{\xi_R}
\newcommand{\xil}{\xi_L}
\newcommand{\bzl}{\ov{\zeta}{}_L}
\newcommand{\bzr}{\ov{\zeta}{}_R}
\newcommand{\zr}{\zeta_R}
\newcommand{\zl}{\zeta_L}
\newcommand{\nbar}{\ov{n}}

\newcommand{\loU}{\lambda_0^{\rm U(1)}}
\newcommand{\llU}{\lambda_1^{\rm U(1)}}
\newcommand{\loN}{\lambda_0^\text{SU($N$)}}
\newcommand{\llN}{\lambda_1^\text{SU($N$)}}
\newcommand{\poU}{\psi_0^{\rm U(1)}}
\newcommand{\plU}{\psi_1^{\rm U(1)}}
\newcommand{\poN}{\psi_0^\text{SU($N$)}}
\newcommand{\plN}{\psi_1^\text{SU($N$)}}

\newcommand{\CPC}{CP($N-1$)$\times$C }
\newcommand{\CPCn}{CP($N-1$)$\times$C}

\newcommand{\MN}{M_\text{SU($N$)}}
\newcommand{\MU}{M_{\rm U(1)}}

\newcommand{\bi}{{\bar \imath}}
\newcommand{\bj}{{\bar \jmath}}
\newcommand{\bk}{{\bar k}}
\newcommand{\bl}{{\bar l}}
\newcommand{\bm}{{\bar m}}

\newcommand{\tgamma}{\wt{\gamma}}
\newcommand{\btgamma}{\ov{\tgamma}}

\begin{document}

%%%%%%%%%%%%%%%%%%%%%%%%%%%%%%%%%%%%%%%%%%%%%%%%%%%%%%%%%%%%%%%%%%%%%%%%%%%%%%%%%%
%
%	  		        S E C T I O N
%
%%%%%%%%%%%%%%%%%%%%%%%%%%%%%%%%%%%%%%%%%%%%%%%%%%%%%%%%%%%%%%%%%%%%%%%%%%%%%%%%%%

\begin{center}

{\Large \bf   Erratum: Description of the Heterotic String \\[2mm]
Solutions in \boldmath{${\rm U}(N)$} SQCD \\[4mm]
[Phys. Rev. D 79, 085015 (2009)]}
\end{center}

\begin{center}
{\bf P. A. Bolokhov$^{a,b}$, M.~Shifman$^{c}$, and \bf A.~Yung$^{c,d,e}$}
\end {center}
\begin{center}

$^a${\it Physics and Astronomy Department, University of Pittsburgh, Pittsburgh, Pennsylvania, 15260, USA}\\
$^b${\it Theoretical Physics Department, St.Petersburg State University, Ulyanovskaya~1, 
	 Peterhof, St.Petersburg, 198504, Russia}\\
$^c${\it  William I. Fine Theoretical Physics Institute,
University of Minnesota,
Minneapolis, MN 55455, USA}\\
$^{d}${\it Petersburg Nuclear Physics Institute, Gatchina, St. Petersburg
188300, Russia}\\
$^e${\it Institute of Theoretical and Experimental Physics, Moscow
117259, Russia}
\end{center}

The notation for the heterotic deformation parameter $ \gamma $ is used inconsistently with Ref.~\cite{SYhet}. 
Throughout the paper $ \gamma $ should be replaced with $ \tgamma $, which is related to the original
$ \gamma $ of Ref.~\cite{SYhet} as
\[
	\tgamma ~~=~~ \sqrt{2/\beta}\, \gamma\,.
\]
The affected places are Eqs. (5.3), (5.4), (5.5), (7.6), the second and
the third next equations after Eq. (5.3) and anywhere in the text where $ \gamma $ is discussed.

%next equation after 
%Eq. (5.3) which determines $ \tgamma $ in terms of parameter $ \delta $,
%the third next equation after Eq. (5.3) which shows the supersymmetry transformations
%of the heterotic CP($N-1$) model.
%and anywhere in the text where $ \gamma $ is discussed.

Formula (5.4) misses factors of $ \sqrt{2} $ in the last two terms in the integrand, and an overall
normalizing factor of $ 2/g_0^2 $. 
The correct formula should read
\setcounter{equation}{3}
\setcounter{section}{5}
\beq
\label{exte}
	\mc{L}_{(0,2)} ~~=~~ \frac{2}{g_0^2}\, \int\, d^4\theta\, \lgr K(\Phi,\ov{\Phi}) 
		~-~ 2\, \ov{\mc{B}}\,\mc{B}  ~+~  \sqrt{2}\,\tgamma\,\mc{B}\,K  
					     ~+~ \sqrt{2}\,\ov{\tgamma}\,\ov{\mc{B}}\,\ov{K} \rgr,
\eeq
where
\[
	\frac{2}{g_0^2} ~~=~~ 2\beta
\]
	defines the coupling constant of the sigma model.

The Lagrangian in Equation (5.5) misses the normalizing factor of $ 2\beta $. 
The correct version should be
\begin{align}
%
\notag
	\frac{\mc{L}_{(0,2)}}{2\beta}
                    & ~~=~~  \bzr\, i\p_L\, \zr 
			~+~ g_{i\bj}\,\p_\mu \phi^i \p_\mu\ov{\phi}{}^\bj
			~+~ \frac{1}{2}\, g_{i\bj}\, \psi^i\, i\overleftrightarrow{\slashed{\nabla}} \ov{\psi}{}^\bj 
	\\[2mm]
%
\label{cpn-1g}
			& 
			~~+~~ \tgamma\, g_{i\bj}\, (i \p_L \ov{\phi}{}^\bj)\, \psi_R^i\, \zr
			~+~ \ov{\tgamma}\, g_{i\bj}\, \ov{\psi}{}_R^\bj (i \p_L \phi^i)\, \bzr
			~+~ |\tgamma|^2\, \bzr\,\zr \cdot ( g_{i\bj}\, \ov{\psi}{}_L^\bj\, \psi_L^i )
	\\[2mm]
%
\notag
			& 
			~~+~~ (1 \!-\! |\tgamma|^2)\, (g_{i\bk}\, \ov{\psi}{}_R^\bk\, \psi_L^i)\,
						     (g_{j\bl}\, \ov{\psi}{}_L^\bl\, \psi_R^j)
			~-~ (g_{i\bk}\, \ov{\psi}{}_R^\bk\, \psi_R^i)\, (g_{j\bl}\, \ov{\psi}{}_L^\bl\, \psi_L^j)~.
\end{align}
To see that this Lagrangian matches the one originally introduced in 
Ref.~\cite{SYhet}, one needs to substitute $ \tgamma = \sqrt{2} g_0 \gamma $ 
and to normalize the kinetic term for $ \zr $ canonically.


%%%%%%%%%%%%%%%%%%%%%%%%%%%%%%%%%%%%%%%%%%%%%%%%%%%%%%%%%%%%%%%%%%%%%%%%%%%%%%%%%%
%%%%%%%%%%%%%%%%%%%%%%%%%%%%%%%%%%%%%%%%%%%%%%%%%%%%%%%%%%%%%%%%%%%%%%%%%%%%%%%%%%
%
%                            B I B L I O G R A P H Y
%
%%%%%%%%%%%%%%%%%%%%%%%%%%%%%%%%%%%%%%%%%%%%%%%%%%%%%%%%%%%%%%%%%%%%%%%%%%%%%%%%%%
%%%%%%%%%%%%%%%%%%%%%%%%%%%%%%%%%%%%%%%%%%%%%%%%%%%%%%%%%%%%%%%%%%%%%%%%%%%%%%%%%%
\small
\begin{thebibliography}{99}
\itemsep -2pt

\bibitem{SYhet}
  M.~Shifman and A.~Yung,
  %``Heterotic Flux Tubes in N=2 SQCD with N=1 Preserving Deformations,''
  Phys.\ Rev.\  D {\bf 77}, 125016 (2008)
  [arXiv:0803.0158 [hep-th]].
  %%CITATION = PHRVA,D77,125016;%%
     
\end{thebibliography}


\end{document}
